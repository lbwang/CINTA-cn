% This text is a supplement of CINTA.
% Nov. 2020
% Author: Libin Wang
% SCNU
% additional use of \usepackage{beamerthemesplit}
\documentclass{beamer}

%%%%%%%%%%%%%%%%%%%%%%%%%%%%%%%%%%%%%%%%%%%%%%%%%%%%%%%%%%%%%%%%%%%%%
\usepackage[BoldFont,SlantFont,CJKnumber]{xeCJK} % 允许使用粗体,斜体以及调用CJKnumb宏包
%\usepackage[CJKbookmarks]{hyperref}

%%Support Chinese in Mac!

\setCJKmainfont[BoldFont=STFangsong, ItalicFont=STKaiti]{STSong}
\setCJKsansfont[BoldFont=STHeiti]{STXihei}
\setCJKmonofont{STFangsong}

%\setCJKmainfont[BoldFont=Microsoft YaHei,ItalicFont=YouYuan]{SimSun} %设置默认中文字体

\setmainfont{Times New Roman} % 设置默认英文衬线字体

\setCJKfamilyfont{ls}{LiSu} %定义字体
%%%%%%%%%%%%%%%%%%%%%%%%%%%%%%%%%%%%%%%%%%%%%%%%%%%%%%%%%%%%%%%%%%%%%

%\usepackage{beamerthemesplit} % new 
\usepackage{beamerthemeshadow}
\usepackage{amsmath, amsfonts, amssymb}

%% Setup for Code
\usepackage{listings}
\lstset{language = Python,
	showstringspaces = false,
	columns = fullflexible,
	numbers = left,
	numberstyle = \tiny,
	frame = single}
%Can't work!
%\usepackage{minted}

%%\newproof{pf}{Proof} 
%\newtheorem{thm}{Theorem}[section]
%\newtheorem{fact}[theorem]{Fact}
\newtheorem{prop}[theorem]{Proposition}
\newtheorem{lem}[theorem]{Lemma}
\newtheorem{defn}[theorem]{Definition}
\newtheorem{cor}[theorem]{Corollary}
\newtheorem{rema}[theorem]{Remark}
\newtheorem{clam}[theorem]{Claim}
\newtheorem{exm}[theorem]{Example}

\newcommand{\Q}{{\mathbb Q}}
\newcommand{\Z}{{\mathbb Z}}
\newcommand{\F}{{\mathbb F}}
\newcommand{\N}{{\mathbb N}}
\newcommand{\R}{{\mathbb R}}

% New commands to keep things tidy.
%\newcommand{\ket}[1]{$\left|#1\right\rangle$}
%\newcommand{\Om}[1]{\small $\omega_{#1}$}
\newcommand{\Grp}[1]{\mathbb{#1}}
\begin{document}
\title{Mathematics for Computer Science--环与域} 
\author{Libin Wang} 
\institute{School of Computer Science, South China Normal University}
\date{\today} 

\frame{\titlepage} 

\frame{\frametitle{Table of contents}\tableofcontents} 

\section{Ring(环)}
\label{sec:ring}
\subsection{定义}

\frame{\frametitle{Ring(环).}
\begin{defn}
环$R$是一个非空集合,在$R$上有两种封闭的二元操作:加法(记为$+: R \times R \mapsto R$)
和乘法(记为$*: R \times R \mapsto R$),并且满足以下条件:
\begin{enumerate}
\item 在加法(+)上$R$是一个阿贝尔群,加法的单位元记为$0$,加法上的逆元记为$-a$;
\item $R$在乘法(*)上满足结合律;
\item 乘法在加法上满足分配律。
\end{enumerate}
\end{defn}
}

\frame{\frametitle{Ring(环).}
具体地表示为公式,对任意$a,b, c\in R$,环$R$满足以下公理:
\begin{eqnarray}
a + (b + c ) = (a + b) + c   \quad \quad \mbox{($+$的结合律)}\\
a + b  =  b + a   \quad \quad \mbox{($+$的交换律)}\\
a + 0 = 0 + a \quad \mbox{($+$的单位元)}\\
a + (- a) = (- a) + a = 0 \quad \quad \mbox{($+$的逆元)}\\
a *(b * c) = (a *b) * c \quad  \quad\mbox{($*$的结合律)}\\
%a * 1= 1*a = a \quad \quad \mbox{($*$的单位元)}\\
(a + b) * c = (a * c) + (b * c) \quad \quad \mbox{(右分配律)}\\
a * (b + c) = (a * b) + (a *c) \quad \quad \mbox{(左分配律)}
\end{eqnarray}
}

\frame{\frametitle{Ring(环).}
\begin{defn}
\begin{itemize}
\item 如果$R$在乘法上也满足交换律,
则称$R$为\emph{交换环},否则称为非交换环。

\item 如果$R$在乘法上具有单位元,则称环$R$为带单位元的环。
\end{itemize}
\end{defn}
}

\frame{\frametitle{Ring实例.}

\begin{example}
\begin{enumerate}
\item $R = \{0\}$是只有一个元素的最小环,称为平凡环或零环。如果一个环中,$1 \neq 0$,则这个环是非平凡环。
一个最小的非平凡环就是$R = \{0, 1\}$,或者记为$\Z_2$。\pause

\item 在普通意义的加法和乘法上,容易验证$\Z, \mathbb{Q}, \mathbb{R}, \mathbb{C}$都是交换环。\pause

\item 整数中所有的偶数在一般的加法与乘法上形成环,同样记为$2\Z$,这是一个不带单位元的环。
           更一般地,对任意整数$n\in \Z$,$n\Z$在一般的加法与乘法上形成环。\pause

\item 对任意正整数$n\in \N$,$\Z_n$是模$n$的加法群。如果给加法群$\Z_n$配置上乘法$*$,
乘法$*$被定义为整数上的模$n$乘法,即$\forall a, b \in \Z_n$,
$a*b \triangleq ab\bmod{n}$。容易验证$\Z_n$是一个交换环。

\end{enumerate}
\end{example}
}

\frame{\frametitle{Ring实例.}

\begin{example}
\begin{enumerate}

\item 取$p$为任意素数,$\Z_p$在模$p$的加法与模$p$的乘法下成环,
而且$\Z_p$的所有非零元素在乘法上都有逆元。以后我们将重点讨论这种特殊的环。\pause

\item 对任意环$R$,$R$的直积$R \times R$形成环,环的加法与乘法定义
为$\forall (a,b), (c,d) \in R \times R$序对,
$(a, b) + (c, d) \triangleq (a+c, b+d)$和$(a, b)  (c, d) \triangleq (ac, bd)$。\pause

\item 实数上的$n \times n$矩阵在普通的矩阵加法和矩阵乘法上形成环,也称为矩阵环,记为$M_n(\mathbb{R})$。这是一种非交换环。
\end{enumerate}
\end{example}
}

\subsection{基本属性}

\frame{\frametitle{Ring的属性.}
\begin{prop}
如果环中包含乘法单位元,则加法交换律必然成立。
\end{prop}
\begin{proof}
设$R$是环,任取$a, b \in R$,考虑$(a + b)(1 + 1) $,分别应用左分配律和右分配律,有:
\[(a + b)(1 + 1) = (a + b) + (a + b) = (a + a) + (b + b)  \]
所以有$a + b = b + a$,即加法交换律成立。
\end{proof}
}

\frame{\frametitle{Ring的属性.}
\begin{prop}
设$R$是一个环,且$a, b \in R$,则有:
\begin{enumerate}
\item $a0 = 0a = 0$
\item $a(-b) =  (-a)b = -ab$
\item $(-a)(-b) = ab$
\end{enumerate}
\end{prop}
}

\frame{\frametitle{Ring的属性.}
\begin{proof}
根据分配律,
\[ a 0 = a  (0 + 0) = a0 + a0\]
则$a0 = 0$,同理$0a = 0$。同样根据分配律
\[ab + a(-b) = a(b + (-b)) = a 0 = 0\]
所以,$-(ab) = a(-b)$,同理$-(ab) = (-a)b$。最后,
根据以上结论,
$(-a)(-b) = -(a(-b)) = -(-(ab)) = ab$。
\end{proof}
}

\subsection{整环与子环}
\frame{\frametitle{整环(Integral Domain.)}
\begin{definition}
给定一个环$R$,对任意非$0$元素$a \in R$,如果存在非$0$元素$b\in R$使得$a*b = 0$,
则称$a$为一个\emph{零因子}(\emph{zero divisor})。如果交换环$R$中没有零因子,
即$\forall a, b \in R$,$a, b \ne 0$,如果$ab = 0$则有$a = 0$或$b = 0$, 
则称$R$为整环。
\end{definition}
}

\frame{\frametitle{整环实例}
\begin{example}
\begin{enumerate}
\item 在普通意义的加法和乘法上,$\Z, \mathbb{Q}, \mathbb{R}, \mathbb{C}$都是整环。比如,考虑$a, b \in \Z$,
$ab = 0$ 当且仅当$a=0$或者$b = 0$。其他实例,验证类似。

\item 可验证偶数环$2\Z$是整环。更一般地,对任意的$n\in \Z$,$n\Z$都是整环。

\item 已知$\Z_n$是一个交换环。但是,$\Z_n$并不是整环。比如,$\Z_{15}$中,$(3*5) \bmod{15} = 0$,
$3$和$5$都是$\Z_{15}$中的零因子。

\item 矩阵环$M_n(\mathbb{R})$不是整环,因为存在$A, B \in M_n(\mathbb{R})$,使得$AB = 0$但是$A$
和$B$都不为$0$。
\end{enumerate}
\end{example}
}

\frame{\frametitle{整环属性}
\begin{prop}
设$D$是一个交换环,$D$是整环当且仅当对任意元素$a,b,c\in D$,且$a \neq 0$,若$ab = ac$,则$b = c$。
\end{prop}
}

\frame{\frametitle{整环属性}
\begin{proof}
\begin{enumerate}
\item $\Rightarrow$. $D$是整环,则$D$中无零因子。若$a \neq 0$且$ab = ac$,则$a(b - c) = 0$,则$b - c = 0$,即$b = c$。

\item $\Leftarrow$. 假设$D$中消去律成立。任取$a, b \in D$且$a \neq 0$。
设$ab = 0$,则有$ab = a0$。根据消去律,
得到$b = 0$。因此,$a$不可能是零因子。
\end{enumerate}
\end{proof}
}

\frame{\frametitle{子环(Subring)}
\begin{definition}
给定环$R$,$R'$是$R$的子集,如果$R'$在环$R$的加法和乘法上也形成环,则称$R'$是$R$的子环,记为$R' \subset R$。
\end{definition}
}

\frame{\frametitle{子环实例}
\begin{example}
\begin{enumerate}
\item 容易验证以下子环序列:$\Z\subset \mathbb{Q}\subset \mathbb{R}\subset \mathbb{C}$。

\item 偶数环是整数环的子环,即$2\Z \subset \Z$。由此可知,子环并不自然继承母环的单位元。
            当然,对任意的$n\in \Z$,有$n \Z \subset \Z$。
\end{enumerate}
\end{example}
}

\frame{\frametitle{子环属性}
\begin{prop}
给定环$R$,$R'$是$R$的子集。$R'$是$R$的子环,当且仅当以下条件满足:
\begin{enumerate}
\item $R' \neq \emptyset$;
\item $\forall a, b \in R'$,有$ab \in R'$;
\item $\forall a, b \in R'$,有$a - b \in R'$。
\end{enumerate}
\end{prop}

\begin{proof}
根据子群命题易得,留作课后练习。
\end{proof}
}

\subsection{环同态、理想与商环}
\frame{\frametitle{环同态与环同构}
\begin{definition}
给定两个环$R$和$R'$,若映射$\phi: R \mapsto R'$满足:$\forall a, b \in R$,
\begin{eqnarray*}
\phi(a + b) = \phi(a) + \phi(b)\\
\phi(ab) = \phi(a)\phi(b)
\end{eqnarray*}
则称$\phi$为一个环同态。如果$\phi$是一个双射,则$\phi$是一个环同构。
环同态$\phi$的Kernel定义为以下集合:
\[\mbox{Ker}\; \phi = \{r \in R: \phi(r) = 0\} \mbox{。}\]
\end{definition}
}

\frame{\frametitle{环同态的性质}
\begin{prop}
设映射$\phi: R \mapsto R'$是环同态,则:
\begin{enumerate}
\item 如果$R$是交换环,则$\phi(R)$也是交换环。

\item 分别记$0$和$0'$是$R$和$R'$的加法单位元,$\phi(0) = 0'$。

\item 分别记$1$和$1'$是$R$和$R'$的乘法单位元,如果$\phi$是满射,则$\phi(1) = 1'$。
\end{enumerate}
\end{prop}
}

\frame{\frametitle{环同态实例--1}
\begin{example}
请验证以下同态实例,体会环同态与群同态的异同点。

\begin{enumerate}
\item 已知$\Z$是环,定义映射$\phi : \Z \mapsto \Z$为$\phi(k) = 2k$,$\forall k \in \Z$,即把所有整数映射到偶数$2\Z$。
已知$\phi$是$\Z$到$\Z$的群同态,但是可以验证$\phi$不是一种环同态。\pause

\item 对任意整数$n\in \Z$,已知$\Z_n$为环,定义映射$\phi : \Z_n \mapsto \Z_n$为$\phi(a) = a^2 \bmod{n}$,$\forall a \in \Z_n$,
即把所有$\Z_n$的元素映射到平方数。可以验证$\phi$不是一种环同态,尽管这种映射在$\Z_n^*$中是一种群同态。

\end{enumerate}
\end{example}
}

\frame{\frametitle{环同态实例--2}
\begin{example}
请验证以下同态实例,体会环同态与群同态的异同点。

\begin{enumerate}

\item 考虑一种特殊的环同态,定义$\phi: 2\Z \mapsto \Z_2$为$\forall k \in 2\Z$,$\phi(k) = k \bmod{2}$。明显,$\phi$
是同态,但不是满同态,它把所有偶数都映射到了$0$。将这种把任意$R$映射到零环$\{0\}$的环同态称为\emph{零同态}。\pause

\item 考虑一种更特殊的环同态,对任意的环$R$,定义映射$\phi: R \mapsto \Z_2$为$\forall a \in R$,$\phi(a) = 0$。可验证,
这是一种零同态,显然不是满同态。特别提醒注意,如果环$R$有乘法单位元$1$,$\phi(1) = 0$。并不是我们期望的映射到$\Z_2$的$1$。

\end{enumerate}
\end{example}
}

\frame{\frametitle{环同态实例--3}
\begin{example}
\begin{enumerate}

\item 对任意环$R$,已知$R\times R$是环。定义映射$\phi: R\times R \mapsto R\times R$为
$\forall (a, b)\in R\times R$,$\phi(a,b) = (a, 0)$。可验证,这是一个环同态,但并非满同态。注意,
在$\phi(R\times R)$中的单位元是$(1,0)$,但是$R\times R$中的单位元是$(1,1)$。


\end{enumerate}
\end{example}
}

\frame{\frametitle{环同态实例--4}
\begin{example}
\begin{enumerate}

\item 设$p$是任意一个奇素数,考虑$2\Z$与$\Z_p$之间的映射$\phi: 2\Z \mapsto \Z_p$,
定义为$\forall k \in 2\Z$,$\phi(k) = k \bmod{p}$。
直观上看,该映射把所有的偶数做模$p$操作满射到环$\Z_p$。容易验证$\phi$是满同态。值得注意的是,
偶数环$2\Z$没有乘法单位元,环$\Z_p$的乘法单位元是$1$,$2\Z$中有无穷多的元素映射到$\Z_p$的乘法单位元$1$上。
也就是说,即使乘法单位元必然映射为乘法单位元,也并非只有乘法单位元才映射为乘法单位元。

\end{enumerate}
\end{example}
}

\frame{\frametitle{理想}
环的理想在群论中的对应概念是正规子群。
\begin{definition}
给定环$R$,$I$是$R$的子环,如果对任意的$r \in R$有$rI \subset I$和$Ir \subset I$,则称$I$是$R$的理想。
\end{definition}\pause
从表面上看,所谓环$R$的理想$I$,首先它是环$R$的子环,其次它具有“\emph{吸收性}”,即对任意的环元素$r\in R$,无论它是否落在$I$中,
用$r$左乘或者右乘$I$,所得到的元素都会落回到$I$中。
}

\frame{\frametitle{理想的实例}
\begin{example}
\label{exm:ideal1}
\begin{enumerate}
\item 所有的环$R$都有两个平凡理想:$\{0\}$和$R$。

\item 如果$R$的理想$I$中包括$1$,则$R=I$。

\item 对任意整数$n\in \Z$,集合$n\Z$是环$\Z$的理想。直观上看,集合$n\Z$包含了所有$n$的倍数,$n\Z$在加法上成群,
而用任意整数乘$n$的倍数还是得到一个$n$的倍数,虽然此时$n\Z$中并不必然有单位元$1$,也不必然有乘法逆元。

\end{enumerate}
\end{example}
}

\frame{\frametitle{“模”理想的同余关系}
\begin{block}{同余关系}
利用理想,可以定义环中元素\emph{模}理想的\emph{同余关系}。设$R$是环,$I$是$R$中的理想,那么对任意的$a, b \in R$,
如果$a \in b + I$,则称$a$和$b$满足以下同余关系:
\[ a \equiv b \pmod{I}\]
或者等价于说,如果$a$与$b$模$I$同余,则存在$i \in I$使得$a = b + i$,即$a - b \in I$。
\end{block}
}

\frame{\frametitle{“模”理想的同余关系}

\begin{example}
已知,$\Z$是环,$2\Z \subset \Z$是$\Z$的理想。那么
\[ 7 \equiv 5 \pmod{2\Z}\]
因为,$7 = 5 + 2$,且$2 \in 2\Z$。但是,
\[ 7 \not\equiv 6 \pmod{2\Z}\]
因为,不存在偶数加$6$会等于$7$。
\end{example}
}

\frame{\frametitle{中国剩余定理--环版本}
\begin{theorem}
设$R$是环,$I$和$J$是$R$中的理想,并且$I + J = R$。 对任意$r_1, r_2 \in R$,以下方程组有解,并且,该方程组的任意解都模$I \cap J$同余。
\begin{eqnarray*}
x &\equiv& r_1 \pmod{I}\\
x &\equiv& r_2 \pmod{J}
\end{eqnarray*}

\end{theorem}
}

\frame{\frametitle{中国剩余定理--环版本}
\begin{proof}
因为$I + J = R$,所以存在$i \in I$和$j \in J$使得$i + j = r_2 - r_1$。注意,此时只需要把$r_2 - r_1$理解为$R$中的某个元素即可。
令$x' = r_1 + i = r_2 - j$,可知$x' \in r_1 + I$且$x' \in r_2 + J$,所以$x'$是方程组的解。

假设方程有两个解$x_1$和$x_2$,那么必然有:
\begin{eqnarray*}
x_1 &\equiv& x_2 \pmod{I}\\
x_1 &\equiv& x_2 \pmod{J}
\end{eqnarray*}
则有$x_1 - x_2 \in I$和$x_1 - x_2 \in J$,因此$x_1 - x_2 \in I \cap J$,即
\[ x_1 \equiv x_2 \pmod{I \cap J}\]

\end{proof}
}

\frame{\frametitle{中国剩余定理--环版本}

\begin{example}
已知,$\Z$是环,$2\Z, 3\Z \subset \Z$是$\Z$的理想,且$2\Z + 3\Z = \Z$。求解:
\begin{eqnarray*}
x &\equiv& 5 \pmod{2\Z}\\
x &\equiv& 4 \pmod{3\Z}
\end{eqnarray*}
易知:$ 7 \equiv 5 \pmod{2\Z}$,且$7 \equiv 4 \pmod{3\Z}$。
$7$是唯一解吗?
\end{example}
}

\frame{\frametitle{主理想(Principal Ideal)}

\begin{prop}
设$R$是一个交换环且有单位元,任取$a \in R$,则集合
\[ \langle a \rangle \triangleq \{ar: r \in R\}\]
是环$R$的一个理想,称之为\emph{主理想}(\emph{Principal Ideal})。
\end{prop}\pause

\begin{alertblock}{主理想 vs 循环群}
环的主理想对应群论中的循环群。
\end{alertblock}
}

\frame{\frametitle{主理想--证明}

\begin{proof}
首先,验证$\langle a \rangle $是非空集合,至少包括$0$和$a$两个元素。
其次,验证$\langle a \rangle$在加法上成群。
最后,验证$\langle a \rangle$ 具有吸收性,即任取$s \in R$乘上$\langle a \rangle $中任意元素$ar$,
必然有
\[s (ar) =  a(sr) \in \langle a \rangle \]
注意,上式成立需要依赖交换律。所以,$\langle a \rangle$是$R$的理想。
\end{proof}

}

\frame{\frametitle{主理想--实例}
\begin{example}
\label{exm:ideal2}
\begin{enumerate}
\item 对任意环$R$,只包含一个元素$0$的主理想$\langle 0 \rangle$称为\emph{零理想}。

\item 对任意带单位元的环$R$,称$\langle 1 \rangle$为\emph{单位理想},显然$R = \langle 1 \rangle$。

\item 对任意整数$n$,集合$n\Z$是整数环$\Z$的理想,也是主理想,$n\Z = \langle n \rangle$。
\end{enumerate}
 
\end{example}
}

\frame{\frametitle{理想与主理想}

\begin{prop}
整数环$\Z$的所有理想都是主理想。
\end{prop}

\begin{proof}
首先,零理想也是主理想,因为$\langle 0 \rangle = \{0\}$。设$I$是整数环$\Z$的一个非零理想,
则$I$中必然包括某些正整数,根据良序原则,
则$I$中必然存在一个最小正整数$n$。对任意的元素$a\in I$,根据除法算法,
存在整数$q$和$r$,$0 \leq r < n$,使得:
\[ a = qn + r\]
也就是,$r = a - qn$,利用理想的属性,可知$r \in I$。又因为$n$是$I$中最小正整数,
所以,$r=0$。因此,$a = qn$,即$I = \langle n \rangle $。
\end{proof}
}

\frame{\frametitle{环同态与Kernel}
\begin{prop}
环同态$\phi: R \mapsto R'$的Kernel是$R$的理想。
\end{prop}

\begin{proof}
根据群论的结论,$K = \mbox{Ker}\;\phi$是$R$的加法子群(并且是正规子群)。
只需要证明$K$具有理想的“吸收性”,
即对任意的$r \in R$和$a \in K$有$ar \in K$和$ra \in K$。
显然如此,因为:
\[\phi(ar) =  \phi(a) \phi(r) = 0 \phi(r) = 0 \]
且
\[\phi(ra) = \phi(r )\phi(a) = \phi(r) 0 = 0\]
\end{proof}
}

\frame{\frametitle{环同态与Kernel}
\begin{example}
对任意整数$n\in \Z$,定义映射$\phi: \Z \mapsto \Z_n$为,对任意$ a \in \Z$,$\phi(a) = a \bmod{n}$。容易验证
这是一个环同态,而$\mbox{Ker}\;\phi$就是$n\Z$。
\end{example}
}

\frame{\frametitle{商环}
要定义商环, 先回顾、理解、熟悉相关的思路。
\begin{block}{第一}
环$R$本身在加法上是阿贝尔群,而其理想$I$则是$R$的正规加法子群,
因此$R/I$在加法上就是一个商群,其中元素就是加法上$I$的陪集。比如,任取$r\in R$,$r + I$
就是是$R/I$中的群元素。为清晰起见,描述$R/I$在加法定义如下。对任意的$r, s \in R$,
\[(r + I) + (s + I) = (r+s) + I\]
\end{block}

}

\frame{\frametitle{商环}

\begin{block}{第二}
$R/I$要形成环,必须定义$R/I$群元素的乘法。无论乘法是什么,根据环的定义,该乘法必须是封闭的,且具有
结合律和分配律。在给出乘法定义之后,这些都需要证明。除此之外,特别强调的是,既然$R/I$的乘法
是对陪集的操作,千万要记得证明良定义属性,因为陪集的代表元不唯一。
\end{block}
}

\frame{\frametitle{商环}
\begin{lemma}
\label{lem:qr_mul}
设$R$是环,$I$是$R$中的理想。商群$R/I$中元素的乘法定义为:
对任意的群元$r, s\in R$,
\[(r + I)(s +I) = rs + I\]
该乘法是一种良定义操作,且具有封闭性、结合律和对加法具有分配律。
\end{lemma}
}

\frame{\frametitle{商环}
\begin{block}{证明乘法是良定义操作的思路}
要证明乘法是一种良定义操作,就是要证明乘法独立于陪集代表元的选择。即证明,如果$r + I = r' + I$,
$s + I = s' + I$,则$(r + I)(s + I) = (r'+ I)(s' + I) $。根据乘法定义,即要证$r s+ I = r' s' + I$。
即证明$r' s' \in rs + I$。另外,
$r + I = r' + I$和$s + I = s' + I$分别意味着$r' \in r + I$,$s' \in s + I$。

\end{block}
}

\frame{\frametitle{商环}
\begin{proof}
要证明乘法是一种良定义操作,即假设$r' \in r + I$,$s' \in s + I$,证明$r's' \in (rs + I)$。
因为$r' \in r + I$,$s' \in s + I$,即存在$i_1, i_2 \in I$使得$r' = r + i_1$和$s' = s + i_2$,因此,
\[ r' s' = (r + i_1) (s + i_2) = rs + r i_2 + i_1 s + i_1 i_2\]
根据理想的吸收性,$r i_2 + i_1 s + i_1 i_2 \in I$,所以,$r's' \in rs + I $。
\end{proof}

\begin{alertblock}{课后练习}
商环乘法的封闭性、结合律和分配律留作课后练习。
\end{alertblock}

}

\frame{\frametitle{商环}
\begin{theorem}
设$R$是环,$I$是$R$中的理想。商群$R/I$在陪集加法与以上引理中定义的乘法上形成环,
称为$R$模$I$的商环,同样记为$R/I$。
\end{theorem}
}

\frame{\frametitle{商环}
\begin{example}
\label{exm:quo_ring}
任取$n \in \Z$,$n\Z = \langle n \rangle$是整数环$\Z$的主理想,则$\Z / n \Z$是商环,其中元素
刚好构成模$n$的完全剩余系。
\end{example}
}

\frame{\frametitle{环的标准同态与第一同构定理}
\begin{definition}
	设$I$是环$R$的理想,定义环同态映射$\phi : R \mapsto R / I$为:对任意$r \in R$,$\phi(r) = r + I$ 。
    并称该映射为环的\emph{标准同态}或者\emph{自然同态},且$\mbox{Ker}\; \phi = I$。 
\end{definition}

\begin{theorem}
[第一同构定理.] 
设$\psi: R \mapsto S$是环同态,记$K= \mbox{Ker\;} \psi$是$R$的理想。如果$\phi: R \mapsto R/K$
是标准同态,则存在唯一同构$\eta: R/K \mapsto \psi(R)$使得$\psi = \eta \phi$。
\end{theorem}
}
\frame{\frametitle{环的标准同态与第一同构定理}
\begin{proof}
根据群论的第一同构定理,在$R$的加法群与$R$模$K$的加法商群之间,
存在唯一的良定义的群同构$\eta: R/K \mapsto \psi(R)$。
该映射定义为,对任意的$r \in R$,有
\[\eta(r + K) = \psi(r)\]
要证明$\eta$是一种环同态,只需要证明,对任意的$r, s \in R$,有
$\eta((r+K)(s+K))= \eta(r+K)\eta(s+K)$。
然而,这是容易的,因为
\begin{eqnarray*}
\eta((r+K)(s+K)) &=& \eta(rs + K)\\
                                  &=& \psi(rs)\\
                                  &=& \psi(r)\psi(s)\\
                                  &=& \eta(r+K)\eta(s+K)                                  
\end{eqnarray*}
\end{proof}
}
\frame{\frametitle{环的标准同态与第一同构定理}
\begin{example}
\label{exm:Zn_mod_nZ}
任取$n \in \Z$,构造映射$\phi: \Z \mapsto \Z_n$为,任取$a \in \Z$,$\phi(a) = a \bmod{n}$。可验证,这是
一个环同态映射,且是满射。$\mbox{Ker}\; \phi = n \Z$,因为所有$n$的倍数都映射为$0$。根据第一同构定理,
$\Z / n \Z \cong \Z_n$。
\end{example}
}

\section{Field (域)}
\subsection{定义与实例}
\frame{\frametitle{域的定义}
\begin{definition}
如果一个带单位元的交换环$R$中的非$0$元素都存在唯一的乘法逆元,即
$\forall a \in R$且$a \neq 0$,则存在唯一的$a^{-1}\in R$使得$a  a^{-1} = a^{-1}a =1$,则称这种代数结构
为域。
\end{definition}
}

\frame{\frametitle{域--实例}
\begin{example}
\begin{enumerate} 
\item 在普通的加法与乘法上,$\mathbb{Q}, \mathbb{R}, \mathbb{C}$都是域。$\Z$不是域,因为乘法上$\Z$不是群。

\item 任意给定素数$p$,在模$p$的加法与乘法上$\Z_p$是域。因为$\Z_p$在加法上是阿贝尔群,而$\Z_p^*$在乘法上
也是阿贝尔群。对任意合数$n\in \Z$,$\Z_n$则不是域,因为$\Z_n - \{ 0\}$乘法上不成群。
\end{enumerate}
\end{example}
}

\frame{\frametitle{域的乘法逆元与零因子}
\begin{prop}[乘法逆元与零因子.]
对任意的域$F$,任取$a, b \in F$,如果$ab = 0$则$a = 0$或者$b = 0$。即域中不存在零因子。
\end{prop}

\begin{proof}
不妨设$a \neq 0$,否则证完。因为$a \neq 0$,则存在$a$的乘法逆元$a^{-1} \in F$且$a^{-1} \neq 0$,使得$a a^{-1} =1$。
等式$ab = 0$两边乘上$a^{-1}$,根据之前的命题,则$b = 0$。
\end{proof}
}

\subsection{整环与域}
\frame{\frametitle{域的乘法逆元与零因子}
\begin{prop}[整环与域.]
每一个有限整环都是域。
\end{prop}

\begin{proof}
证明的思路就是利用有限整环的性质,为每一个非$0$元素找到乘法逆元。
设$D$是一个有限整环,记$D^*$为环中所有非$0$元素的集合。对任意的$a \in D^*$,
构造映射$\lambda_a : D^* \mapsto D^*$为$\lambda_a(d) = ad$,$\forall d \in D^*$。
首先,证明这确实是合理的映射,因为如果$a \neq 0$,$d \neq 0$,则$ad \neq 0$。
然后,因为$D^*$是有限集且$\lambda_a$是从$D^*$到$D^*$的单射,所以$\lambda_a$必然是满射。
因此,必然存在某个$d\in D^*$使得$ad = 1$,又因为$D$是交换环,所以这个$d$就是$a$的乘法逆元。
结论:可为$D$中每一个非零元素都找到乘法逆元,所以$D$是一个域。
\end{proof}
}

\frame{\frametitle{域的乘法逆元与零因子}
\begin{alertblock}{注意.}
由以上证明,读者能体会出为什么要求整环是交换环吗?
\end{alertblock}
}

\subsection{特征}
\frame{\frametitle{特征 (Characteristic)}
\begin{definition}[特征.]
环$R$的\emph{特征}(\emph{characteristic})
定义为最小的正整数$n$使得对任意的$r\in R$,$\underbrace{r + r + \cdots + r}_{n\mbox{个}} = nr = 0 $。
如果不存在这样的$n$,则$R$的特征定义为$0$。
\end{definition}

\begin{example}
\label{exm:char}
\begin{enumerate} 
\item 环$\mathbb{Q}, \mathbb{R}, \mathbb{C}$的特征都是$0$。

\item 对任意素数$p$,域$\Z_p$的特征是$p$。因为$\Z_p$加法群的阶为$p$,
即对任意的$a \in \Z_p$,$pa = 0$。
\end{enumerate}
\end{example}
}

\frame{\frametitle{特征的属性}
\begin{lemma}
设$R$为环,若$1$在加法群的阶为$n$,则$R$的特征为$n$。
\end{lemma}

\begin{proof}
若$1$在加法群的阶为$n$,则$n$是最小正整数使得$n1 = 0$。那么,任取$r\in R$,有:
\[n r = n(1 r) = (n1)r = 0r = 0\]
即$n$是$R$的特征。
\end{proof}
}

\frame{\frametitle{特征的属性}
\begin{prop}
整环的特征或者为素数,或者为$0$。
\end{prop}

\begin{proof}
设整环$D$的特征为$n$,且$n \neq 0$。
如果$n$不是素数,则$n = a b$,且$1 < a, b < n$。
根据以上引理,有
\[ 0 = n 1 = (ab)1 = (a1)(b1)\]
因为$D$是整环,$D$中无零因子,所以必然$a1 = 0$或者$b1=0$。
但是这都意味着$D$的特征小于$n$,矛盾。
\end{proof}
}

\frame{\frametitle{域的特征与阶的关系}
\begin{prop}[有限域的特征.]
阶为$n$的有限域$F$的特征是一个素数$p$,且$p\mid n$。
\end{prop}

\begin{proof}
因为有限域$F$的阶为$n$,且$F$是加法群,
所以对任意的$a\in F$,有$na = 0$。
所以,$F$的特征必然是素数$p$,且$p \mid n$。
\end{proof}
}

\frame{\frametitle{有限域的阶}
以下不加证明给出另一个重要结论。
\begin{prop}[有限域的阶]
如果有限域$F$的特征是素数$p$,则$F$的阶是$p^n$,$n$是某个正整数。
进一步,对任意的素数$p$和正整数$n$,存在阶为$p^n$的有限域,并且
所有的$p^n$阶有限域都同构。
\end{prop}

\begin{example}
$p^n$阶的有限域也记为$\mbox{GF}(p^n)$,$\mbox{GF}$是Galois Field的缩写。
\end{example}
}

\subsection{域的理想}
\frame{\frametitle{域的理想}
\begin{prop}[域的理想]
任何一个域$F$的理想只有$0$和自己本身$F$。
\end{prop}
\begin{proof}
首先,已知$0$和$F$都是$F$的理想。
设$I$是域$F$的非$0$理想,则存在非零元素$a \in I$。因为$F$是域,则存在$a$的乘法逆元$a^{-1} \in F$。
根据理想的吸收性,$a^{-1}a = 1 \in I$。包含$1$的理想$I$等于$F$,即$I = F$。
\end{proof}
}

\frame{\frametitle{域同态是单射}
\begin{prop}[域同态是单射.]
任何一个域同态或者是单射或者是零同态。
\end{prop}

\begin{proof}
域$F_1$到域$F_2$的域同态$\phi$是单射,当且仅当$\mbox{Ker}\; \phi = \{0\} \subset F_1$。
因为$F_1$的理想只有$0$和$F_1$本身,所以当$\mbox{Ker}\; \phi = \{0\} $时$\phi$是单射,
而当$\mbox{Ker}\; \phi = F_1$时,$\phi$是零同态。
\end{proof}
}

\frame{\frametitle{极大理想与素理想}
\begin{definition}[极大理想.]
设$R$是环,$M$是$R$的真子集且是$R$的理想,则称$M$是$R$的\emph{真理想}。
设$M$是$R$的真理想,
如果$M$不是$R$的任意真理想的真子集,则称
$M$是\emph{极大理想}(\emph{maximal ideal})。
即如果$M$是$R$的极大理想,则对$R$的任意理想$I$,若$M \subset I$,则$I = R$。
\end{definition}
\pause

\begin{definition}[素理想.]
    设$P$是交换环$R$的真理想,如果对任意$ab \in P$,则或者$a \in P$,或者$b \in P$,
    就称$P$为\emph{素理想}(\emph{Prime Ideal})。
\end{definition}
}


\frame{\frametitle{极大理想与素理想}
\begin{example}
素理想的“素”确实有“素数”的意味。
设$p$为素数,如果$p \mid ab$,则$p \mid a$或$p \mid b$。请体会素理想定义中要求与之类似之处:
对任意$ab \in P$,则$a \in P$或$b \in P$。令$P=p \Z$,$p$是任意素数,$a \in P$当且仅当$p \mid a$。
所以,从整数的角度上看,素理想确实是素数的倍数形成的理想。当然,理想不能仅停留于此,还需要进一步的抽象,
但是这个例子告诉我们,抽象代数的“抽象”并非凭空而出,往往源自于具体的实例。
\end{example}
}

\frame{\frametitle{极大理想与素理想}
\begin{example}
设$P = \{0, 2, 4, 6\}$为环$\Z_8$的理想,可验证$P$是极大理想,也是素理想。
任取素数$p$,则$p \Z$是$\Z$的素理想。
\end{example}
}

\frame{\frametitle{极大理想与域}
\begin{theorem}[极大理想与域.]
设$R$是交换环,$M$是$R$的理想,则$M$是$R$的极大理想,当且仅当$R/M$是域。
\end{theorem}
}

\frame{\frametitle{极大理想与域}
\begin{proof}
\begin{enumerate}
\item $\Rightarrow.$ 由条件可知$R/M$是交换环,只需要证明$R/M$中所有非零元都有乘法逆元,则$R/M$是域。%即不包括0 + M
定义环同态$\phi: R \mapsto R/M$为$\phi(r) = r + M$,$\forall r \in R$。任取$r \in R$且$r \not \in M$,构造主理想$\langle r + M \rangle$。
因为$\phi$是环同态,所以$I = \phi^{-1}(\langle r + M \rangle)$是$R$中的理想。并且$M$是$I$的真子集,因为$r\in I$但是$r\not \in M$。
又因为$M$是极大理想,所以$I = R$。因此有:$$\phi(1) = 1 + M \in \langle r + M \rangle$$
即存在$s \in R$使得$1 + M = (r + M) (s + M)$。因此,$r + M$有乘法逆元。
%以上证明源自这里。
%https://ttjones.wordpress.com/2011/08/29/rm-is-a-field-iff-m-is-maximal/ 
\end{enumerate}
\end{proof}
}

\frame{\frametitle{极大理想与域}
\begin{proof}
\begin{enumerate}

\item $\Leftarrow.$ 因为$R/M$是域,所以它至少包含两个元素$0 + M$和$1 + M$,即$M$是$R$的真理想。假设存在$R$的理想$I$,且$M$是$I$的真子集,
只需证$I = R$,则可证明$M$是$R$的极大理想。取$a \in I$且$a \not \in M$,可知$a + M$是域$R/M$的非零元素。因此存在$b + M \in R/M$使得:
\[(a +M)(b +M) = ab +M = 1 + M \]
即存在$m \in M$使得$ ab + m = 1$,根据环的封闭性和理想的“吸收性”,可知$1 \in I$。所以,$I = R$。
\end{enumerate}
\end{proof}
}

\frame{\frametitle{极大理想与域}
\begin{example}
任取素数$p$,已知$p \Z$是$\Z$的素理想。因为$\Z / p\Z \cong \Z_p$,$\Z_p$是域,所以$p \Z$是$\Z$的极大理想,
\end{example}
}

\frame{\frametitle{素理想与整环}
\begin{theorem}[素理想与整环.]
设$R$是交换环,$P$是$R$的理想,则$P$是$R$的素理想,当且仅当$R/P$是整环。
\end{theorem}
}

\frame{\frametitle{素理想与整环}
\begin{proof}
\begin{enumerate}
\item $\Rightarrow$. 设$P$是素理想,若$R/P$中的两个元素使得:
\[(a + P) (b + P) = ab + P = 0 + P = P\]
可知,$ab \in P$。不失一般性,若$a \not \in P$,则根据素理想的定义,有$b \in P$。因此,有$b + P = 0 + P$。即$R/P$是整环。

\item $\Leftarrow$. 设$P$是$R$的理想,且$R/P$是整环。假定$ab \in P$。则有:
\[(a + P) (b + P) = ab + P = 0 + P = P\]
根据整环属性,则或者$a + P = P$成立,或者$b + P = P$成立。这意味着,或者$a \in P$或者$b \in P$。说明$P$必须是素理想。
\end{enumerate}
\end{proof}
}

\frame{\frametitle{素理想与整环}
\begin{example}
设$p$是素数,则$p \Z$是$\Z$的理想。可知,$p \Z$是$\Z$的极大理想,因为$\Z / p \Z \cong \Z_p$是域。
\end{example}
}

\frame{\frametitle{极大理想与素理想}
\begin{cor}[极大理想与素理想.]
交换环的每一个极大理想都是素理想。
\end{cor}

\begin{proof}
容易。因为,所有的域都是整环。
\end{proof}

}

\end{document}