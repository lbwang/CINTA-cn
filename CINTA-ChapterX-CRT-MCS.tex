% This text is a supplement of CINTA.
% May. 2018
% Author: Libin Wang
% SCNU
% additional use of \usepackage{beamerthemesplit}
\documentclass{beamer}

%%%%%%%%%%%%%%%%%%%%%%%%%%%%%%%%%%%%%%%%%%%%%%%%%%%%%%%%%%%%%%%%%%%%%
\usepackage[BoldFont,SlantFont,CJKnumber]{xeCJK} % 允许使用粗体,斜体以及调用CJKnumb宏包
%\usepackage[CJKbookmarks]{hyperref}

%%Support Chinese in Mac!

\setCJKmainfont[BoldFont=STFangsong, ItalicFont=STKaiti]{STSong}
\setCJKsansfont[BoldFont=STHeiti]{STXihei}
\setCJKmonofont{STFangsong}

%\setCJKmainfont[BoldFont=Microsoft YaHei,ItalicFont=YouYuan]{SimSun} %设置默认中文字体

\setmainfont{Times New Roman} % 设置默认英文衬线字体

\setCJKfamilyfont{ls}{LiSu} %定义字体
%%%%%%%%%%%%%%%%%%%%%%%%%%%%%%%%%%%%%%%%%%%%%%%%%%%%%%%%%%%%%%%%%%%%%

%\usepackage{beamerthemesplit} % new 
\usepackage{beamerthemeshadow}
\usepackage{amsmath, amsfonts, amssymb}

%% Setup for Code
\usepackage{listings}
\lstset{language = Python,
	showstringspaces = false,
	columns = fullflexible,
	numbers = left,
	numberstyle = \tiny,
	frame = single}
%Can't work!
%\usepackage{minted}

%%\newproof{pf}{Proof} 
%\newtheorem{thm}{Theorem}[section]
%\newtheorem{fact}[theorem]{Fact}
\newtheorem{prop}[theorem]{Proposition}
\newtheorem{lem}[theorem]{Lemma}
\newtheorem{defn}[theorem]{Definition}
\newtheorem{cor}[theorem]{Corollary}
\newtheorem{rema}[theorem]{Remark}
\newtheorem{clam}[theorem]{Claim}
\newtheorem{exm}[theorem]{Example}
\begin{document}
%\title{A Concrete Introduction to Number Theory and Algebra--CRT} 
\title{Mathematics for Computer Science -- CRT} 
\author{Libin Wang} 
\institute{School of Computer Science, South China Normal University}
\date{\today} 

\frame{\titlepage} 

%\frame{\frametitle{Table of contents}\tableofcontents} 


\section{The Chinese Remainder Theorem(中国剩余定理)} 

\frame{\frametitle{导引.} 
Chinese Remainder Theorem(中国剩余定理),或称为\emph{中国余数定理}则更准确。

\begin{block}{历史.}
中国剩余定理有着满满的中国传统文化元素,与之相关的名人包括:春秋时期的孙子、秦汉时期的韩信、宋朝的沈括和秦久韶等。
南北朝时期(公元5世纪)的数学著作《孙子算经》卷下第二十六题所谓“物不知数”问题则为国人所熟知:
有物不知其数,三三数之剩二,五五数之剩三,七七数之剩二。问物几何?
\end{block}
}

\frame{\frametitle{Motivation.} 
中国剩余定理
讨论一元同余方程组的高效解法。
	\begin{example}{Example 1.}
	Suppose we wish to solve:
	\[x \equiv  2 \pmod{5}\]
	
	\[x \equiv 3 \pmod{7}\]
	\end{example} \pause

\begin{block}{Solution.}
	1. We have $x = 5t + 2$ from the first congruence, for $t \in \mathbb{N}$;\pause
	
	2. Substitute for $x$ in the second congruence, $ 5t + 2 \equiv 3 \; (\mbox{mod} \; 7)$;\pause
	
	3. Simplifies, get $5t \equiv 1 \; (\mbox{mod} \; 7)$;\pause
	
	4. Multiplies both sides with $5^{-1}$ to get $t = 7s + 3$ for $s \in \mathbb{N}$;\pause
	
	5. Finally, $x = 35s + 17$, means $x \equiv 17 \; (\mbox{mod} \; 35)$.
	
\end{block}
}

\frame{\frametitle{The Chinese Remainder Theorem--CRT.} 
	For any system of equations like this, the \emph{Chinese Remainder Theorem}, short for CRT, tells us there is always a unique solution up to a certain modulus, and describes how to find the solution efficiently.
	\begin{theorem}
	     Let $p, q$ be primes, $n = pq$. For each $a \in \mathbb{Z}_p$, $b \in \mathbb{Z}_q$, there is unique $x$, $0 \leq x < n$ such that
	     $x \equiv a\; (\mbox{mod} \; p)$ and $x \equiv b\; (\mbox{mod} \; q)$.
	\end{theorem}
}

\frame{\frametitle{The Chinese Remainder Theorem--CRT.} 
	\begin{theorem}
		Let $p, q$ be coprime positive integers, $n = pq$. For each $a \in \mathbb{Z}_p$, $b \in \mathbb{Z}_q$, there is a unique $x$, $0 \leq x < n$ such that
		$x \equiv a\; (\mbox{mod} \; p)$ and $x \equiv b\; (\mbox{mod} \; q)$.
	\end{theorem}
Proof Idea
\begin{enumerate}
\item Given $a$ and $p$, how can we find some $1_p$ s.t. $a * 1_p \equiv a \pmod{p}$? \pause

\item $1_p$ must be a $1$ under modulo $p$\pause

\item Recall something from our previous courses, we have many options. \pause

\item We try this: find some $c$ s.t. $1_p \equiv c c^{-1} \equiv 1 \pmod{p}$ \pause

\item Similarly, for the $q$-part, we find $1_q \equiv d d^{-1} \equiv 1 \pmod{q}$ \pause

\item And then $x$ must be something like that $x = (a * 1_p + b * 1_q) = (a c c^{-1} + b d d^{-1})$, what should be $c$ and $d$?
\end{enumerate}
}

\frame{\frametitle{The Chinese Remainder Theorem--CRT.} 
	\begin{theorem}
		Let $p, q$ be coprime positive integers, $n = pq$. For each $a \in \mathbb{Z}_p$, $b \in \mathbb{Z}_q$, there is a unique $x$, $0 \leq x < n$ such that
		$x \equiv a \pmod{p}$ and $x \equiv b \pmod{q}$.
	\end{theorem}

\begin{proof}
	By construction. Since $p, q$ are coprime, these must exist $p_1$ and $q_1$ such that $p_1 \equiv p^{-1}  \pmod{q}$ and $q_1 \equiv q^{-1} \pmod{p}$. Let integer $x$ be:
	$$ y = a q q_1 + b p p_1 $$
	It is easy to check that $y$ satisfies both equations.
	It remains to show no other solutions exist modulo $n$. Suppose $\exists z \neq y$ is another solution. Then $(z - y ) = tp$ and $(z - y) = sq$, for some $t, s \in \mathbb{N}$. Since $p$ and $q$ are coprime, then $(z - y) = kpq$, for $k \in \mathbb{N}$. Hence $z \equiv y \; (\mbox{mod} \; n)$. 
	
\end{proof}
}

\frame{\frametitle{Example.} 
	\begin{example}{Example 2.}
		Suppose we wish to solve:
		$$x \equiv  2\; (\mbox{mod} \; 5)$$
		$$x \equiv 3 \; (\mbox{mod} \; 7)$$
	\end{example}

\begin{block}{Solution.}
	1. Let $a=2$, $b = 3$, $p = 5$, $q = 7$, $n = pq = 35$;
	
	2. Compute $p_1 \equiv p^{-1} (\mbox{mod} \; q)$ and $q_1 \equiv q^{-1} (\mbox{mod} \; p)$ using EGCD algorithm; $p_1 = 3 $, $q_1 = 3$;
	
	3. $ y \equiv a q q_1 + b p p_1 \; (\mbox{mod} \; n)$; $y = 17$;
	
	4. It is easy to check that $y$ is a correct solution.	
\end{block}
}

\frame{\frametitle{Exercise.} 
	\begin{block}{求解以下方程.}
		Suppose we wish to solve:
		$$x \equiv  3 \pmod{11}$$
		$$x \equiv 4 \pmod{13} $$
	\end{block}
\pause
\begin{block}{Ans.}
	...
\end{block}
}

\frame{\frametitle{The Chinese Remainder Theorem--CRT--另一种证明.} 

\begin{proof}
	By construction. Since $p, q$ are coprime, these must exist $p_1$ and $q_1$ such that $p p_1 + q q_1 = 1$, by Bezout Theorem. Then we have:
	\begin{eqnarray*}
		p p_1 &\equiv& 1 \pmod{q}\\
		q q_1 &\equiv& 1 \pmod{p}
	\end{eqnarray*}
	Let $t = a q q_1 + b p p_1$, we can check that, $t$ is indeed a solution of the equations. 
\end{proof}
}

\frame{\frametitle{The Chinese Remainder Theorem--CRT--两种证明的对比.} 

\begin{block}{Remark.}
	第一种证明有点啰嗦,但是它告诉你,如何去发现证明。第二种证明简洁,直观非常棒,但是你不容易想到。或者,你不知道如何去发现这种证明。后者看上去很容易,但是,如果你意识到,这种证明竟然不出现在大多数的数论教材,你就明白,它是一种颇为独到的见解。然而它也有缺点,从它出发难以找到“通解”,也就是说它不具备第一种证明的那种“思想”。
\end{block}
}

\frame{\frametitle{Generization.} 
	For Several Equations, we have a generized version of CRT.
\begin{theorem}
	Let $m_1, m_2, \cdots, m_n$ be a set of pairwise relatively prime integers. Then the system of $n$ equations:
	\begin{eqnarray*}
	  x & \equiv &  a_1\; (\mbox{mod} \; m_1)\\
	  & \cdots & \\
	  x & \equiv & a_n \; (\mbox{mod} \; m_n)
	\end{eqnarray*}
	has a unique solution for $x$ modulo $M$ where $M = m_1 m_2 \cdots m_n$.
\end{theorem}	

}

\frame{\frametitle{Generization.} 
	\begin{proof}
		By construction. Let $M= \prod_{i = 1}^{n} m_i$, $b_i = M/m_i$, $b_i' = b_i ^{-1} \; (\mbox{mod} \; m_i)$. Then 
		$$y = \sum_{i=1}^{n} a_i b_i b_i' \; (\mbox{mod} \; M)$$
		is the unique solution.
	\end{proof}
}

\frame{\frametitle{A perspective from Abstract Algebra.} 
	\begin{block}{Motivation.}
		Let $n = pq$, $p , q > 1$ are relatively prime. Given a positive integer $x$, it can be expressed as a unique pair $( [x \; \mbox{mod} \; p], [x \; \mbox{mod} \; q ])$.
	\end{block}
}

\frame{\frametitle{A perspective from Abstract Algebra.} 
\begin{theorem}
	Let $p, q > 1$ be coprime, $n = pq$. Then
	$$\mathbb{Z}_n \cong \mathbb{Z}_p \times \mathbb{Z}_q \quad \mbox{and} \quad \mathbb{Z}_n^* \cong \mathbb{Z}_p^* \times \mathbb{Z}_q^* .$$
\end{theorem}

\begin{proof}
1. Define $\phi$ as a function mapping from $\mathbb{Z}_n$ to $\mathbb{Z}_p \times \mathbb{Z}_q$ as:
$$\phi(x) \triangleq ([x \bmod{p}], [x \bmod{q}])$$

2. Show $\phi$ is bijective.

3. Check that $\phi(x)$ preserves the group operation.\pause

The proof that it is an isomorphism from $\mathbb{Z}_n^*$ to $\mathbb{Z}_p^* \times \mathbb{Z}_q^*$ is similar.
\end{proof}
}

\frame{\frametitle{Remark.}
\begin{rema}
首先,注意到$\mathbb{Z}_p \times \mathbb{Z}_q$是加法群,请验证!
\end{rema}
}

\frame{\frametitle{How to prove the bijection?}
\begin{proof}
证明映射$\phi$是一种双射,即证明$\phi$是满射且单射。满射显然,因为根据
中国剩余定理,任意序对中的两个同余式在模$n$下存在唯一解。证明单射即证明,
如果对任意正整数$a, b < n$,有
$([a \bmod{p}], [a \bmod{q}]) = ([b \bmod{p}], [b \bmod{q}])$,则$a = b$。再次根据
中国剩余定理可得。
\end{proof}
} 

\frame{\frametitle{How to prove the isomorphism?}
\begin{proof}
证明映射$\phi$保持群操作,即需要证明:
\begin{eqnarray*}
\phi(a+b) &=& ([(a+b) \bmod{p}], [(a+b)\bmod{q}]) \\
                     &=&  ([a \bmod{p}], [a \bmod{q}]) + ([b \bmod{p}], [b\bmod{q}])\\
                     &=&  \phi(a) + \phi(b)
\end{eqnarray*}
\end{proof}
}

\frame{\frametitle{Example.} 
	\begin{example}{Example 3.}
		Take $n = 15 = 5 \cdot 3$.
		$\mathbb{Z}_n^* = \{ 1, 2, 4, 7, 8, 11, 13, 14\}$ is isomorphic to $\mathbb{Z}_5^* \times \mathbb{Z}_3^*$ since we can give following correspondence:
		$$ 1 \leftrightarrow (1, 1) \quad 2 \leftrightarrow(2,2) \quad 4 \leftrightarrow (4,1) \quad 7 \leftrightarrow (2,1)$$
		$$8 \leftrightarrow (3,2) \quad 11\leftrightarrow(1,2) \quad 13 \leftrightarrow (3,1) \quad 14 \leftrightarrow (4,2)$$
	\end{example}
}

\frame{\frametitle{Example.} 
	\begin{example}{Example 4.}
		To compute $14\cdot 13 \; \mbox{mod} \;15$. Since $14 \leftrightarrow (4,2)$ and $13 \leftrightarrow (3,1)$, we have:
		$$(4, 2) \cdot (3, 1) = ([4\cdot 3 \; \mbox{mod} \; 5], [2\cdot 1 \; \mbox{mod} \; 3]) = (2,2).$$
		Note that $(2, 2) \leftrightarrow 2$, which is the correct answer.
	\end{example}
}

\frame{\frametitle{Example.} 
	\begin{example}{Example 5.}
		To compute $11^{53} \; \mbox{mod} \;15$. Since $11 \leftrightarrow (1,2)$ and $2 \equiv -1  \; \mbox{mod} \; 3$ we have:
		$$(1, 2)^{53} = ([1^{53} \; \mbox{mod} \; 5], [-1^{53} \; \mbox{mod} \; 3]) = (1, -1 \; \mbox{mod} \; 3) = (1, 2).$$
	 Thus, $11^{53} \; \mbox{mod} \; 15 = 11$ 
	\end{example}
}

\frame{\frametitle{Example.} 
\begin{example}{Example 6.}
\label{exm:eq_sqr_cong}
	设$n = p q$为合数,$p$和$q$是两个不同的素数。以下等式在模$n$的意义上有多少个解?
	\[ x^2 \equiv 1 \pmod{n}\]
为此,需要解以下两个同余方程
	\begin{eqnarray*}
	x^2 \equiv 1 \pmod{p} \\
	x^2 \equiv 1 \pmod{q}
	\end{eqnarray*}
以上两式分别有两个不同的解。因此在模$n$的意义上总共有$4$个解。
同时,这也就解决了上一章的一个遗留问题。
\end{example}
}

\frame{\frametitle{Example.} 
	\begin{example}{Example 7.}
		To compute $12^{12} \bmod{143}$ \pause
		解:
		\begin{itemize}
		\item 记$n = 143$,$n = p * q = 13 * 11$;
		
		\item 求:$12^{12} \bmod{13}$ 和 $12^{12} \bmod{11}$;
		
		\item 所以,$12^{12} \bmod{143} = 1$.
		\end{itemize}
	\end{example}
	
}


\frame{\frametitle{Little thought.} 
	\begin{rema}
		A practical application: if we have many computations to perform on $x \in \mathbb{Z}_n^*$
		(e.g. RSA signing and decryption), we can convert $x$ to $(a, b) \in \mathbb{Z}_p^* \times \mathbb{Z}_q^*$
		and do all the computations on $a$ and $b$
		instead before converting back. 
		
		This is often cheaper because for many algorithms, doubling the size of the input more than doubles the running time.
	\end{rema}
}


\frame{\frametitle{Homeworks Exercises.} 
	\begin{block}{Homeworks.}
		1. Using CRT to solve:
		$$x \equiv 8\; (\mbox{mod} \; 11)$$
		$$x \equiv 3\; (\mbox{mod} \; 19)$$
		
		2. Using CRT to solve the system of congruence:
			\begin{eqnarray*}
			x & \equiv & 1\; (\mbox{mod} \; 5)\\
			x & \equiv & 2\; (\mbox{mod} \; 7)\\
			x & \equiv & 3\; (\mbox{mod} \; 9)\\
			x & \equiv & 4\; (\mbox{mod} \; 11)
		\end{eqnarray*}
		% ans = 1731
		3. Write a program(C or Python) to solve CRT.	
	\end{block}

}

\frame{\frametitle{Homeworks Exercises.} 
	\begin{block}{Homeworks.}
		4. Complete the proof that  it is an isomorphism from $\mathbb{Z}_n^*$ to $\mathbb{Z}_p^* \times \mathbb{Z}_q^*$.
		
		5. Let $p = 5$ and $q = 7$, $n = pq$. Please explicitly give the correspondece between $\mathbb{Z}_n^*$ and $\mathbb{Z}_p^* \times \mathbb{Z}_q^*$. Hint: Programming is permitted.
	\end{block}
	
}

\frame{\frametitle{Homeworks Exercises.} 
	\begin{block}{Homeworks.}
		6. 请求解$x^{113} \equiv 15 \pmod{221}$[提示:这是一道似曾相识的习题,然而中国剩余定理能保证你顺利完成目标。]
		
	\end{block}
	
}
\end{document}

